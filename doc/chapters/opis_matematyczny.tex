\section{Opis matematyczny}
\label{sec_opis_mat}

\subsection{Twierdzenie Bayesa}
\label{subsec_tw_bayesa}
Twierdzenie w teorii prawdopodobieństwa określające zależność między prawdopodobieństwem warunkowym wystąpienia zdarzeń $A|B$ i $B|A$. Przyjmijmy zbiór zdarzeń $X$, w którym zdarzenia $B_i \in X$, $P(B_i)>0$ ${(i=1,2, \dots , n)}$ tworzą układ zupełny (iloczyn każdych dwóch zdarzeń jest zdarzeniem niemożliwym, natomiast suma wszystkich zdarzeń jest zdarzeniem pewnym). Wówczas dla dowolnego $A \in X$ zachodzi następująca zależność:

	\begin{equation}
	\label{eq_tw_bayesa_1}
		P(B_i | A) = \frac{P(A | B_i) P(B_i)}{P(A)}
	\end{equation}

Wykorzystując dodatkowo wzór na prawdopodobieństwo całkowite, powyższa zależność może zostać przekształcona do następującej postaci:

	\begin{equation}
	\label{eq_tw_bayesa_2}
		P(B_i | A) = \frac{P(A | B_i) P(B_i)}{\sum_{k=1}^{n} P(A | B_k)P(B_k)}
	\end{equation}


\subsection{Model probabilistyczny}
\label{subsec_model}

Zdefiniujmy k-elementowy zbiór klas $C = {\{C_1, C_2, \dots, C_k\}}$ oraz dane do klasyfikacji opisane jako wektor ${x = \{x_1, x_2, \dots, x_n\}}$ zawierający $n$ niezależnych cech. Prawdopodobieństwo przynależności do danej klasy może zostać zapisane z wykorzystaniem twierdzenia Bayesa (rozdział~\ref{subsec_tw_bayesa}):

	\begin{equation}
	\label{eq_p_ck_x}
		P(C_i | x) = \frac{P(x | C_i) P(C_i)}{P(x)}
	\end{equation}

Prawdopodobieństwo $P(x)$ występujące w mianowniku wzoru (\ref{eq_p_ck_x}) nie zależy od $C$ i jest stałe, licznik może natomiast zostać przekształcony poprzez wykorzystanie definicji prawdopodobieństwa warunkowego:

	\begin{equation}
	\label{eq_p_numerator1}
		P(x_1, \dots , x_n | C_i)P(C_i) = P(x_1, \dots, x_n, C_i)
	\end{equation}

	\begin{equation}
	\label{eq_p_numerator2}
		P(x, C_i) = P(x_1 | x_2 , \dots , x_n, C_i)P(x_2 | x_3, \dots , x_n, 	C_i) \dots P(x_{n-1} | x_n, C_i)P(x_n | C_i)P(C_i)
	\end{equation}
	
Wykorzystując przyjęte na początku założenie, że cechy ${x_1, \dots, x_n}$ są niezależne można wyprowadzić następującą zależność:

	\begin{equation}
	\label{eq_p_numerator_simp1}
		P(x_j | x_{j+1}, \dots , x_n, C_i) = P(x_j | C_i) \quad dla \quad j = \{1, 2, \dots, n-1\}
	\end{equation}

Podstawiając (\ref{eq_p_numerator_simp1}) do równania (\ref{eq_p_numerator2}) otrzymujemy:
	
	\begin{equation}
	\label{eq_p_numerator_simp2}
		P(x_1, \dots, x_n, C_i) = P(C_i) \prod_{j = 1}^n P(x_j | C_i)
	\end{equation}

Ostatecznie wzór (\ref{eq_p_ck_x}) można zapisać w postaci:

	\begin{equation}
	\label{eq_p_ck_x_simp}
		P(C_i | x) = \frac{P(C_i) \prod_{j = 1}^n P(x_j | C_i)}{P(x)}
	\end{equation}


\subsection{Rozkłady prawdopodobieństwa}
\label{sec_rozklady}

Występujące w równaniu (\ref{eq_p_ck_x_simp}) prawdopodobieństwo, że dany obiekt należy do klasy $j$ na podstawie cechy $i$ może zostać wyznaczone na różne sposoby, w zależności od analizowanego problemu. Jak zostało napisane w rozdziale \ref{subsec_bayes_detekcja_pulsu}, w niniejszym projekcie wykorzystywany jest rozkład normalny. Warto pamiętać, że istnieje także możliwość wykorzystania innych funkcji gęstości prawdopodobieństwa. 

\subsubsection{Rozkład normalny}
\label{subsec_gauss}

	\begin{equation}
	\label{eq_gauss}
	p(x_i | C_j) = \frac{1}{\sigma \sqrt{2 \pi}}\exp\bigg(\frac{-(x-\mu_{ij})^2}{2\sigma_{ij}^2}\bigg), \quad
	-\infty < x < \infty, \; -\infty < \mu_{ij} < \infty, \; \sigma_{ij}  > 0
	\end{equation}

\noindent Gdzie: $\mu_{ij}$ -- średnia, $\sigma_{ij}$ -- odchylenie standardowe

\subsubsection{Rozkład lognormalny}
\label{subsec_lognorm}

	\begin{equation}
	\label{eq_lognorm}
	p(x_i | C_j) = \frac{1}{x \sigma_{ij} (2 \pi)^{1/2}} \exp\bigg( \frac{-(log(x/m_{ij}))^2}{2 \sigma_{ij}^2} \bigg), \quad
	0 < x < \infty, \; m_{ij} > 0, \; \sigma_{ij}  > 0
	\end{equation}

\noindent Gdzie: $m_{ij}$ -- parametr skali, $\sigma_{ij}$ -- parametr kształtu

\subsubsection{Rozkład Gamma}
\label{subsec_gamma}

	\begin{equation}
	\label{eq_gamma}
	p(x_i | C_j) = \frac{(x/b_{ij})^{c_{ij}-1}}{b_{ij} \Gamma(c_{ij})} \exp \bigg( \frac{-x}{b_{ij}} \bigg), \quad
	0 \leq x < \infty, \; b_{ij} > 0, \; c_{ij}  > 0
	\end{equation}

\noindent Gdzie: $b_{ij}$ -- parametr skali, $c_{ij}$ -- parametr kształtu
	
\subsubsection{Rozkład Poissona}
\label{subsec_gamma}

	\begin{equation}
	\label{eq_poisson}
	p(x_i | C_j) = \frac{\lambda_{ij}^x \exp \big( -\lambda_{ij} \big)}{x!}, \quad
	0 \leq x < \infty, \; \lambda_{ij} > 0, \: x=0,1,2, \dots
	\end{equation}

\noindent Gdzie: $\lambda_{ij}$ -- średnia