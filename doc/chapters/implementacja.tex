\section{Implementacja}
\label{sec_implementacja}

\subsection{Prototyp}
\label{subsec_prototyp}

Prototyp algorytmu został zaimplementowany w \textit{Pythonie}. Ta implementacja miała na celu dostosowanie opisu matematycznego zamieszczonego w rozdziale \ref{sec_opis_mat} do problemu klasyfikacji pulsu i wstępne zweryfikowanie poprawności metody oraz oszacowanie dokładności przed przejściem do właściwej implementacji w \textit{C++}. 

Oprócz realizacji samego algorytmu dodano także wiele funkcji upraszczających i pozwalających częściowo zautomatyzować procedurę weryfikacji i testowania. Program daje możliwość wykorzystania jednego zbioru wejściowego, który zostaje automatycznie w losowy sposób rozdzielony na uczący (70\%) i testowy (30\%). Drugą opcją jest wykorzystanie dwóch osobnych zestawów danych wejściowych, jednego tylko do uczenia i drugiego do testów. Ponad to istnieje także możliwość douczenia modelu z wykorzystaniem kolejnych plików wejściowych. Po załadowaniu danych, program przeprowadza proces generacji modelu prawdopodobieństwa (inaczej proces uczenia), a następnie uruchamia procedurę testową i zwraca dokładność metody, swoistość, czułość oraz opcjonalnie czas wykonania.

Może się okazać, że nie wszystkie z 18 cech przedstawionych w rozdziale \ref{sec_wstep} mają wpływ na klasyfikację z wykorzystaniem metody Naiwnego Bayesa. W związku z tym dodano także możliwość wyłączenia z modelu prawdopodobieństwa niektórych cech opisujących \textit{QRS}. Takie podejście pozwoliło przetestować wiele przypadków. Dokładny opis i zestawienie przeprowadzonych testów zostało przedstawione w rozdziale \ref{sec_testy}, natomiast szczegółową instrukcję do programu umieszczono w rozdziale \ref{dodatekA}.


\subsection{Implementacja w C++}
\label{subsec_implementacja_cpp}

\subsubsection{Opis}
Implementacja w \textit{C++} miała na celu zapewnienie wyższej wydajności w stosunku do implementacji prototypu w \textit{Pythonie}. Sam algorytm został dokładnie odwzorowany bez stosowania żadnych uproszczeń, wszystkie dodatkowe funkcje oraz sposób użycia programu pozostały bez zmian. 

Jedynym mankamentem jest odczyt danych wejściowych z pliku. \textit{Python} udostępnia zoptymalizowane funkcje do parsowania danych zapisanych w formacie \textit{csv}. W przypadku \textit{C++} konieczne było napisanie od podstaw funkcji odczytującej. Takie rozwiązanie powoduje, że sam odczyt i przetwarzanie danych z pliku trwa dłużej niż w przypadku prototypu. W związku z~tym, aby nie zaburzać rzeczywistego czasu wykonywania się algorytmu zdecydowano się pomiar czasu rozpoczynać w momencie zakończenia odczytu danych (oczywiście w obu wersjach implementacji).

\subsubsection{Wykorzystane biblioteki}
W celu polepszenia wydajności implementacji. poprawienie funkcjonalności i uproszczenia kodu źródłowego wykorzystano opensourcowe biblioteki. Z racji niewielkich rozmiarów zostały one dołączone do folderu projektowego, takie rozwiązanie ułatwia kompilację i przenoszenie projektu.

Przy implementacji algorytmu zastosowana została bilbioteka \textit{Eigen}\cite{eigen}. Z jej wykorzystaniem można zoptymalizować operacje matematyczne wykonywane na dużych wektorach danych. W omawianym przypadku zbiór danych wejściowych zawierał do 30 tysięcy rekordów, zastosowanie zoptymalizowanej biblioteki pozwoliło znacząco podnieść wydajność programu a także uprościć implementację.

W celu implementacji obsługi dodatkowych funkcji z poziomu wiersza poleceń wykorzystano bibliotekę \textit{Tclap} \cite{tclap}. Udostępnia ona prosty interfejs do obsługi argumentów z wiersza poleceń. Jej zaletą jest implementacja jedynie z wykorzystaniem plików nagłówkowych, dzięki czemu jest łatwa w użyciu i integracji z różnymi projektami.

