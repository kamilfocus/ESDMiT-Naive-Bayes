\section{Dodatek A: Instrukcja uruchomienia programów}
\label{dodatekA}

Repozytorium z projektem dostępne jest pod linkiem: \href{https://github.com/kamilfocus/ESDMiT-Naive-Bayes}{Klasyfikacja pulsu - Naiwny Bayes}. 

\subsection{Program w C++}
\label{subsec_cpp_instr}

\subsubsection{Kompilacja}
Program uruchamiany jest przez wiersz poleceń, kod źródłowy znajduje się w katalogu \texttt{Program}. Wszystkie dodatkowe biblioteki umieszczone są w folderze projektowym, nie ma konieczności dołączania zewnętrznych bibliotek. Do kompilacji wykorzystywany jest plik \texttt{Makefile}, generujący plik wykonywalny \texttt{bayes.exe}. Program po każdym uruchomieniu generuje tymczasowy plik \texttt{bayes.dump} z ostatnim zapamiętanym modelem prawdopodobieństwa. 

\subsubsection{Dane wejściowe}
Tak jak w przypadku prototypu konieczne jest określenie dwóch plików, ścieżki mogą zostać podane jako argumenty w wierszu poleceń. W tym celu należy wykorzystać flagę \textit{--d} w~następujący sposób: \textit{--d $<$ścieżka do qrs\_data$>$ --d $<$ścieżka do class\_id$>$}. W przypadku niezdefiniowania ścieżek zostanie wykorzystane domyślne wejście.

\subsubsection{Lista cech}
Wykorzystując flagę \textit{--f} można zdefiniować maskę binarną aktywującą poszczególne cechy. Lista cech zawartych w danych testowych jest zgodna z z analizą przedstawioną w rozdziale \ref{sec_wstep} i przedstawia się następująco:\\
\textit{r\_peak, r\_peak\_value, rr\_pre\_interval, rr\_post\_interval, 
p\_onset, p\_onset\_val, p\_peak, p\_peak\_val, p\_end, p\_end\_val, 
qrs\_onset, qrs\_onset\_val, qrs\_end, qrs\_end\_val, t\_peak, t\_peak\_val, 
t\_end, \\t\_end\_val}.\\
Maska podawana jest w postaci heksadecymalnej z przedrostkiem \textit{0x}. Najmłodszy bit odpowiada pierwszej cesze z listy, przykładowo maska: \textit{--f $0x0040c$} powoduje, że algorytm bierze pod uwagę jedynie: \textit{rr\_pre\_interval, rr\_post\_interval, qrs\_onset}.

\subsubsection{Pozostałe funkcje}

Pozostałe funkcje mogą zostać aktywowane poprzez ustawienie odpowiednich flag:
\begin{itemize}
\item{-r -- resetuje efekty uczenia zapisane w pliku}
\item{-v -- rozszerzone logowanie, wyświetla dodatkowo efekty uczenia w postaci wartości średniej i odchylenia standardowego każdej cechy oraz liczbę próbek wykorzystanych do nauczenia danej klasy.}
\item{-l -- wykonywany jest jedynie proces uczenia/douczenia z wykorzystaniem podanego zbioru wejściowego}
\item{-t -- wykonywane są jedynie testy z wykorzystaniem podanego zbioru wejściowego, model probabilistyczny musi zostać wygenerowany wcześniej z użyciem flagi \textit{--l} i jest odczytywany z pliku} 
\item{-m -- mierzy i wyświetla czas wykonania programu}
\end{itemize}

Jedyna zabroniona kombinacja to użycie jednocześnie flag \textit{--l} i \textit{--t}. W celu wykorzystania tylko jednego zbioru i automatycznego rozdzielenia danych na zbiór na testowy oraz uczący należy pominąć obie flagi. Dokładny opis korzystania programu został zamieszczony w rozdziale \ref{dodatekA}

\subsubsection{Przykład użycia}
Kolejność podawania flag przy uruchomieniu programu jest dowolna, przykładowe wywołanie programu:\\ 
\texttt{bayes --d ../ReferencyjneDane/100/train\_data.txt --d ../ReferencyjneD\\ane/101/train\_label.txt --r --l --f $0x3fffe$}\\ 
Instrukcja ta powoduje usunięcie modelu prawdopodobieństwa zapisanego w pliku (jeżeli istnieje) i wygenerowanie nowego (bez przeprowadzenia testów z powodu flagi \textit{--l}). Zgodnie ze zdefiniowaną maską, algorytm uwzględni wszystkie cechy oprócz pierwszej (\textit{r\_peak}). 


\subsection{Prototyp}
\label{subsec_prototyp_instr}
Model programowy algorytmu został napisany przy pomocy Pythona. Do skonfigurowania środowiska uruchomieniowego dla projektu służą poniższe instrukcje:

\begin{enumerate}
	\item Kod programu jest kompatybilny z interpreterem języka Python w wersji 2.7.x i znajduje się w folderze \texttt{Model}.
	\item Aby włączyć program, należy uruchomić skrypt \texttt{main.py} z odpowiednimi argumentami, składnia jest identyczna jak w przypadku programu w \textit{C++}.
	\item Wynik działania programu jest przekierowany na standardowe wyjście oraz zapisany do pliku \texttt{bayes\_logger.txt}.
\end{enumerate}

Program po każdym uruchomieniu generuje tymczasowy plik \texttt{bayes.dump} z ostatnim zapamiętanym modelem prawdopodobieństwa. Przykład użycia:\\
\texttt{bayes --d ../ReferencyjneDane/101/test\_data.txt --d ../ReferencyjneD\\ane/101/test\_label.txt --t}\\ 
Instrukcja ta powoduje przeprowadzenie testów z wykorzystaniem podanych plików wejściowych, model prawdopodobieństwa zostanie odczytany z tymczasowego pliku wygenerowanego w trakcie poprzedniego wywołania programu.

\subsection{Dodatkowe skrypty}
\label{subsec_skrypty_instr}

