\section{Testy i wnioski}
\label{sec_testy}

\subsection{Dane testowe i metodologia}
Do testów zaimplementowanej metody wykorzystano dane referencyjne zamieszczone w repozytorium \cite{ref_data_repo}. Wspomniany zbiór zawiera 38 zestawów danych wejściowych o różnych długościach i złożoności (ilości obecnych klas). Użyto danych zamieszczonych w katalogu \\ \texttt{data\_unique\_labels}, zostały one tak przygotowane, że w każdym z nich klasa oznaczająca zdrową osobę ma to samo id, takie podejście zdecydowanie ułatwia procedurę testowania. Dodatkowo każdy ze zbiorów został rozdzielony na uczący (70\%) i testowy(30\%). 

Po przeprowadzeniu szczegółowej analizy z wykorzystaniem dodatkowych skryptów i narzędzi opisanych w rozdziale \ref{subsec_narzedzia} zdecydowano się przeprowadzić testy algorytmu wykorzystującego wszystkie dostępne cech zespołu \textit{QRS}, ponieważ właśnie taka konfiguracja zapewnia najlepsze wyniki.

W celu sprawdzenia dokładności przygotowanej metody wykorzystano następujące wskaźniki wskaźniki:

\begin{enumerate}
\item \textit{skuteczność} definiowana jako procent poprawnie rozpoznanych klas
\item \textit{czułość} -- stosunek wyników prawidłowo rozpoznanych jako pozytywnych (prawdziwie dodatnich) do sumy wyników prawdziwie dodatnich i błędnie rozpoznanych jak fałszywe (fałszywie ujemnych), w naszym przypadku przez wynik pozytywnie dodani rozumiane jest poprawne rozpoznanie osoby zdrowej, natomiast negatywnie ujemny to zakwalifikowanie osobny zdrowej do grypy chorych

\item \textit{swoistość} -- stosunek wyników prawidłowo rozpoznanych jako negatywne (pozytywnie ujemnych) do sumy wyników pozytywnie ujemnych i błędnie rozpoznanych jako pozytywne (fałszywie dodatnich), w naszym przypadku przez wynik pozytywnie ujemny rozumiane jest poprawne rozpoznanie osoby chorej, natomiast negatywnie dodatni to zakwalifikowanie osobny chorej do grypy zdrowych
\end{enumerate} 

\noindent Zarówno \textit{czułość} i \textit{swoistość} są wskaźnikami binarnymi, ponieważ w analizowanych zbiorach klasa 1 zawsze oznacza osobę zdrową przyjęto założenie, że klasyfikacja do jakiejkolwiek innej klasy oznacza osobę chorą.

\subsection{Wyniki}
Przed przystąpieniem do analizy dokładności algorytmu przeprowadzono porównanie wydajności obu implementacji. Czas wykonania obu wersji dla poszczególnych zbiorów testowych zamieszczono w tabeli \ref{tab:czasy}. Zgodnie z oczekiwaniami implementacja w \textit{C++} w każdym przypadku jest szybsza od prototypu. W zależności od testowanego zbioru czas wykonania jest nawet do trzech razy szybszy.

Drugim etapem była analiza dokładności algorytmu, szczegółowe wyniki dla poszczególnych zbiorów zostały zamieszczone w tabeli \ref{tab:skutecznosc}.
W przypadku niektórych testów w kolumnach \textit{swoistość} i \textit{czułość} można zauważyć wartości \texttt{NaN}, oznacza to że podczas obliczania danego wskaźnika wystąpiło dzielenie przez zero. Taka sytuacja może wystąpić w przypadku gdy w danych zbiorze nie ma w ogóle osób chorych, wtedy nie odnotowujemy żadnego przypadku pozytywnie ujemnego ani fałszywie dodatniego przez co podczas obliczania swoistość dzielimy przez 0.  Analogicznie sytuacja może pojawić się podczas obliczania czułości gdy zbiór zawiera jedynie zdrowe osoby.

\begin{table}
	\centering
	\begin{tabular}{|c|r|r|r|r|}
		\hline
		$Plik$ & Python [s] & C++ [s] \\
		\hline
		100 & 0.032      & 0.015      \\
		\hline
		101 & 0.017      & 0.011      \\
		\hline
		102 & 0.024      & 0.012      \\
		\hline
		103 & 0.015      & 0.006      \\
		\hline
		104 & 0.013      & 0.007      \\
		\hline
		105 & 0.056      & 0.030      \\
		\hline
		106 & 0.032      & 0.013      \\
		\hline
		108 & 0.003      & 0.002      \\
		\hline
		109 & 0.019      & 0.011      \\
		\hline
		111 & 0.009      & 0.004      \\
		\hline
		112 & 0.042      & 0.016      \\
		\hline
		113 & 0.023      & 0.012      \\
		\hline
		118 & 0.004      & 0.002      \\
		\hline
		119 & 0.023      & 0.011      \\
		\hline
		121 & 0.015      & 0.006      \\
		\hline
		122 & 0.030      & 0.011      \\
		\hline
		124 & 0.030      & 0.017      \\
		\hline
		200 & 0.009      & 0.006      \\
		\hline
		201 & 0.003      & 0.002      \\
		\hline
		202 & 0.022      & 0.010      \\
		\hline
		203 & 0.033      & 0.015      \\
		\hline
		205 & 0.059      & 0.026      \\
		\hline
		208 & 0.050      & 0.020      \\
		\hline
		209 & 0.012      & 0.005      \\
		\hline
		210 & 0.054      & 0.037      \\
		\hline
		212 & 0.022      & 0.008      \\
		\hline
		213 & 0.113      & 0.053      \\
		\hline
		214 & 0.021      & 0.007      \\
		\hline
		215 & 0.019      & 0.009      \\
		\hline
		217 & 0.011      & 0.005      \\
		\hline
		219 & 0.055      & 0.024      \\
		\hline
		221 & 0.013      & 0.006      \\
		\hline
		222 & 0.057      & 0.032      \\
		\hline
		223 & 0.005      & 0.004      \\
		\hline
		228 & 0.033      & 0.019      \\
		\hline
		231 & 0.046      & 0.030      \\
		\hline
		233 & 0.053      & 0.023      \\
		\hline
		234 & 0.070      & 0.033      \\
		\hline
		Średnia & 0.030      & 0.015      \\
        \hline
	\end{tabular}
	\caption{Porównanie czasu wykonania implementacji Bayes dla Pythona i C++}
	\label{tab:czasy}	
\end{table}

\begin{table}
	\centering
	\begin{tabular}{|c|r|r|r|r|}
		\hline
		$Plik$ & Skuteczność [\%] & Czułość [\%] & Specyficzność [\%] \\
		\hline
		100 & 100.00     & 100.00     & 100.00     \\
		\hline
		101 & 100.00     & 100.00     & NaN        \\
		\hline
		102 & 95.05      & NaN        & 95.05      \\
		\hline
		103 & 100.00     & 100.00     & NaN        \\
		\hline
		104 & 84.38      & 18.75      & 97.50      \\
		\hline
		105 & 99.59      & 99.79      & 50.00      \\
		\hline
		106 & 98.09      & 98.42      & 96.72      \\
		\hline
		108 & 96.88      & 100.00     & 0.00       \\
		\hline
		109 & 98.83      & NaN        & 98.83      \\
		\hline
		111 & 100.00     & NaN        & 100.00     \\
		\hline
		112 & 100.00     & 100.00     & NaN        \\
		\hline
		113 & 99.56      & 100.00     & 0.00       \\
		\hline
		118 & 87.10      & NaN        & 87.10      \\
		\hline
		119 & 100.00     & 100.00     & 100.00     \\
		\hline
		121 & 100.00     & 100.00     & NaN        \\
		\hline
		122 & 100.00     & 100.00     & NaN        \\
		\hline
		124 & 99.53      & NaN        & 99.53      \\
		\hline
		200 & 79.17      & 80.88      & 50.00      \\
		\hline
		201 & 51.72      & 100.00     & 0.00       \\
		\hline
		202 & 96.23      & 96.23      & NaN        \\
		\hline
		203 & 93.66      & 93.66      & NaN        \\
		\hline
		205 & 99.76      & 100.00     & 0.00       \\
		\hline
		208 & 86.52      & 86.64      & 85.00      \\
		\hline
		209 & 94.92      & 95.41      & 88.89      \\
		\hline
		210 & 99.64      & 100.00     & 0.00       \\
		\hline
		212 & 97.92      & 100.00     & 95.56      \\
		\hline
		213 & 88.53      & 91.90      & 51.61      \\
		\hline
		214 & 90.91      & NaN        & 90.91      \\
		\hline
		215 & 99.45      & 99.44      & 100.00     \\
		\hline
		217 & 81.82      & 97.78      & 47.62      \\
		\hline
		219 & 89.02      & 89.24      & 0.00       \\
		\hline
		221 & 100.00     & 100.00     & 100.00     \\
		\hline
		222 & 75.67      & 75.50      & 76.67      \\
		\hline
		223 & 96.88      & NaN        & 96.88      \\
		\hline
		228 & 98.35      & 98.86      & 95.00      \\
		\hline
		231 & 98.69      & 94.00      & 99.40      \\
		\hline
		233 & 96.60      & 96.80      & 85.71      \\
		\hline
		234 & 100.00     & 100.00     & NaN        \\
		\hline
		Średnia & 94.06      & 93.98      & 69.60      \\
        \hline
	\end{tabular}
	\caption{Wyniki klasyfikacji pulsu za pomocą Naiwnego Bayesa}
	\label{tab:skutecznosc}	
\end{table}