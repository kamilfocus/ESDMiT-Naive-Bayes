\section{Algorytm}
\label{sec_algorytm}

\subsection{Założenia}
\label{sec_zalozenia}

Jak zostało wspomniane we wstępie, Naiwny Klasyfikator Bayesa zakłada wzajemną niezależność poszczególnych cech. Pierwszym etapem jest proces uczenia klasyfikatora. W tym celu definiuje się zbiór uczący, zawierający wektory cech (w naszym przypadku są to wektory cech zespołu QRS) oraz odpowiadającą im klasę. Na podstawie zbioru uczącego tworzony jest rozkład prawdopodobieństwa, służący do klasyfikacji nowych próbek. 

\subsection{Klasyfikacja}
\label{sec_klasyfikacja}

\subsection{Zbiór testowy i uczący}
\label{sec_test_ucz}